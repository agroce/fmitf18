%%%%%%%%%%%%%%%%%%%%%%%%%%%%%%%%%%%%%%%%%%%%%%%%%%%%%%%%%%%%%
\begin{center}
{\Large\sf\textbf{Project Summary\\\fulltitle{}}}
\end{center}

Most large or critical software projects devote a significant
fraction of their effort to testing; tests are the primary products of
this effort, and are ubiquitous in real-world development.  The
research community has developed numerous techniques to automatically
generate tests.  Unfortunately, once tests are created, whether by
manual effort or automated techniques, they are among the most inert
of computational objects.  At present, tests are stored, executed, and
occasionally read by humans, but they are essentially treated as
givens, amenable neither to modification (other than by hand),
combination, separation, or amplification.  For example, while there
is a large body of research studying how to minimize, prioritize, and
select tests from a test suite, it is only in the last three years
that work has finally appeared that modifies individual tests within
the suite in the pursuit of these goals.  Tests are, at present, not
first-class entities: they support only a few, limited, operations,
lacking even such basic capabilities as composition and decomposition.

\paragraph{Proposed Research}

This project will investigate a practical theory of tests as entities
that support operations such as composition, decomposition,
normalization, and generalization. The two basic prongs of the
research are: (1) how to represent tests so as to enable rich
operations and (2) the development of algorithms for useful operations
on tests.  This research will be grounded in a set of realistic
problems that require a better understanding of tests as first-class
entities.

For example, a primary limitation of human-produced tests is that
humans are not prolific test authors, nor are they typically
imaginative enough to exercise the more obscure behaviors of a
software system. However, automatically generated tests typically
cannot check for complex properties of system output, because human
understanding is needed to evaluate correctness in the absence of a
complete formal specification.  A sophisticated ability to compose
tests should enable the ``amplification'' (in terms of coverage and
complexity) of hand-crafted tests with automatically generated
additional behavior that preserves the effectiveness of the
human-produced checks on correct behavior.  Effective test composition
would also make it possible to automatically produce integration/system tests for
systems, even heterogeneous ones, where the individual elements have
effective tests but the interactions between the sub-systems are untested.

The ability to automatically decompose tests into smaller tests, each
checking correctness of a smaller aspect of system behavior, such that
the set of all such tests retains the power of the original test has
many potential benefits:  it can improve seeded test generation;
reduce the high cost of regression testing by enabling a much finer-grained
approach to selection, minimization, and prioritization of tests;
reduce the chance of ``flaky'' test behavior; and support
recently-proposed test-based approaches to self-adaptive software.
Such tests are also likely to be easier to understand and
use in debugging.


\paragraph{Intellectual Merit}

While tests are common objects of academic study, the very concept of
a test is poorly defined.  The commonalities and differences between
types of tests (unit tests or API-call sequences vs. file inputs, for
example) are poorly understood.  The approach of this project is
rooted in a novel conception of tests: tests can be understood as
functions without inputs that return a test status indicating
failure(s).  Additionally, tests typically feature a linear (but
possibly hierarchical) notion of components and a set of causal
properties that define the reason for a given test's existence, and
control its output given the state of the program under test.  The
idea of tests as programs of a restricted form should also enable the
adaption of techniques from test analysis to more general program
analysis, and vice versa.

\paragraph{Broader Impacts}

Improvements in testing for software systems will have a direct impact
on the quality of systems critical to the economy, national defense,
scientific research in other fields, and the quality of life for
information consumers.  Inadequate testing methods cost tens of
billions of dollars each year, despite test budgets that often consume
more than half of development costs.  More effective test generation
and debugging is critical to addressing this problem.

\paragraph{Key Words:}
software testing, regression testing, test generation, debugging


%%%%%%%%%%%%%%%%%%%%%%%%%%%%%%%%%%%%%%%%%%%%%%%%%%%%%%%%%%%%%
