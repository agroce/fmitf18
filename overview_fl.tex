\documentclass[11pt]{article}
% -----------------------------------
\usepackage[utf8]{inputenc}
\usepackage{fullpage}
\usepackage[numbers]{natbib}
\usepackage{doi}
\usepackage{graphicx}
\usepackage{enumitem}
\usepackage{times}
\usepackage[T1]{fontenc}
% -----------------------------------
\title{Principles and tools for the analysis of IoT communication
  protocols and their implementations applied to the\\Southwest
  Experimental Garden Array}
% -----------------------------------
\begin{document}
\maketitle

It is notoriously difficult to design correct and secure communication
protocols. One of the most famous example is the Needham Schroeder
Public Key protocol~\cite{NS1978:CACM}: it took 18 years to discover a
flaw in this protocol~\cite{LOW1996:TACAS}, and it was done using
formal methods. The {\em implementation} of a correct protocol also
matters as the Heartbleed vulnerability in the OpenSSL implementation
shows.

In the case of Heartbleed, the problem was a C runtime error: an
access to an invalid memory region, and it was due to an implicit
assumption on the input of a function that was actually false.
Combination of static and dynamic analyses can detect such
vulnerabilities~\cite{KKP2015:HVC} because they are not due to complex
interactions related to the protocol.

The aim of this proposal is to:
\begin{itemize}[itemsep=0pt]
\item study the principles of the analysis of communication protocols and
  their implementations,
\item implement the theoretical principles in two sets of tools,
\item use these tools to statically and dynamically analyze a network
  of devices deployed in the Southwest Experimental Garden
  Array~\cite{SEGA}, a common garden instrument for examining climate
  change, genetic and environmental factors affecting plants and
  associated communities.
\end{itemize}

\paragraph{Principles} Assuming a communication protocol is described
as timed automata~\cite{AD1994:TCS} or probabilistics timed automata,
that are satisfy some timed temporal logic
formulas~\cite{BLM2017:LNCS}, and it is implemented as a set of
imperative programs, the two main questions are:
\begin{itemize}[itemsep=0pt]
\item Given the timed automata and a set of programs supposed to
  implement them, how to annotate the programs to be able to check
  they correctly implement the protocol, or to find bugs?
\item Given a set of programs, possibly with annotations, supposed to
  implement a protocol, how to generate the protocol as timed automata
  or probabilistic timed automata amendable to formal verification
  with respect to timed temporal logic formulas?
\end{itemize}
In this first part, we will consider a toy imperative language with
usual control structures, unbounded integers, addresses, and
non-recursive procedures.

\paragraph{Tools} In this part we will consider the C programming
language, Frama-C~\cite{KKP2015:FAC} and DeepState~\cite{DEEPSTATE}
for the analysis of programs, and
Uppaal~\footnote{\url{http://www.uppaal.org}} and
Prism~\cite{KNP2011:CAV}) for the analysis of protocols. We main
questions are:
\begin{itemize}[itemsep=0pt]
\item How to handle the C constructs that are not in the toy language?
\item How to extend the specification languages of Frama-C and
  DeepState?
\item How to handle intra-program parallelism?
\item How to automate the methods as much as possible?
\item \ldots
\end{itemize}

\paragraph{Application} SEGA

% ------------------------------------
\bibliography{bibliography_fl}
%\bibliography{references}
\bibliographystyle{plainnat}
\end{document}
