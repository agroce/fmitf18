\newcommand{\bs}{\ensuremath{\backslash}}

%===========================================================================
\section{Runtime Verification using E-ACSL}
%===========================================================================

%------------------------------------

\begin{frame}{Objectives of E-ACSL}

  \begin{itemize}
  \item Frama-C initially designed as a static analysis platform
  \item Extended with plugins for \blue{dynamic analysis}
  \item E-ACSL: runtime assertion checking tool
    \begin{itemize}
  	\item detect runtime errors 
  	\item detect annotation failures
  	\item treat a concrete program run (i.e. concrete inputs)
  \end{itemize}  
  \end{itemize}
  
\end{frame}

% -------------------------------------------------------------------

\begin{frame}[fragile]{E-ACSL plugin at a Glance}

%  \vspace{-5mm}
  \begin{center}
    \Large{\red{\url{http://frama-c.com/eacsl.html}}}
  \end{center}

\begin{itemize}
\item convert \blue{E-ACSL annotations into C code}
\item implemented as a \blue{Frama-C} plugin
\end{itemize}

\begin{center}
\begin{tikzpicture}
  \node (init) at (0,0) [anchor=north]
{\small \begin{lstlisting}[escapechar=|,frame=none]
int div(int x, int y) {
  |\red{/*@ assert y-1 != 0; */}|
  return x / (y-1);
}
\end{lstlisting} };

  \node (eacsl) at (6,0) [anchor=north]
{\small \begin{lstlisting}[escapechar=|,frame=none]
int div(int x, int y) {
  |\red{/*@ assert y-1 != 0; */}|
  |\blue{e\_acsl\_assert(y-1 != 0);}|
  return x / (y-1);
}
\end{lstlisting} };

\draw [->,color=red] (init) to node [auto] { \footnotesize{E-ACSL} }
(eacsl); 
\end{tikzpicture}
\end{center}
\pause
\begin{itemize}
%% \item \red{\texttt{\bs{}result}} requires to introduce
%%   extra-variables
%% \item \red{\texttt{\bs{}at(x,L)}} requires to introduce code at L
\item the general translation is \blue{more complex than it may look}
\end{itemize}

\end{frame}

