\makeatletter

%%%%%%%%%%%%%%%%%%%%%%%%%%%%%%%%%%%%%%%%%%%%%%%%%%%%%%%%%%%%%%%%%%%%%%%%%%%%%% 
% 
% \btIfInRange{number}{range list}{TRUE}{FALSE}
% 
% Test if int number <number> is element of a (comma separated) list of ranges
% (such as: {1,3-5,7,10-12,14}) and processes <TRUE> or <FALSE> respectively

\newcount\bt@rangea
\newcount\bt@rangeb

\newcommand\btIfInRange[2]{%
  \global\let\bt@inrange\@secondoftwo%
  \edef\bt@rangelist{#2}%
  \foreach \range in \bt@rangelist {%
    \afterassignment\bt@getrangeb%
    \bt@rangea=0\range\relax%
    \pgfmathtruncatemacro\result{ ( #1 >= \bt@rangea) && (#1 <= \bt@rangeb) }%
    \ifnum\result=1\relax%
    \breakforeach%
    \global\let\bt@inrange\@firstoftwo%
    \fi%
  }%
  \bt@inrange%
}
\newcommand\bt@getrangeb{%
  \@ifnextchar\relax%
  {\bt@rangeb=\bt@rangea}%
  {\@getrangeb}%
}
\def\@getrangeb-#1\relax{%
  \ifx\relax#1\relax%
  \bt@rangeb=100000%   \maxdimen is too large for pgfmath
  \else%
  \bt@rangeb=#1\relax%
  \fi%
}

%%%%%%%%%%%%%%%%%%%%%%%%%%%%%%%%%%%%%%%%%%%%%%%%%%%%%%%%%%%%%%%%%%%%%%%%%%%%%% 
% 
% \btLstHL<overlay spec>{range list}
% 
\newcommand<>{\btLstHL}[1]{%
  \only#2{\btIfInRange{\value{lstnumber}}{#1}%
    {\color{blue!20}}%
    {\def\lst@linebgrdcmd####1####2####3{}}% define as no-op
  }%
}%


%%%%%%%%%%%%%%%%%%%%%%%%%%%%%%%%%%%%%%%%%%%%%%%%%%%%%%%%%%%%%%%%%%%%%%%%%%%%%% 
% 
% \btInputEmph<overlay spec>[listing options]{range list}{file name}
% 
\newcommand<>{\btLstInputEmph}[3][\empty]{%                                                    
  \only#4{%
    \lstset{linebackgroundcolor=\btLstHL{#2}}%
    \lstinputlisting{#3}%
  }% \only
}


%%%%%%%%%%%%%%%%%%%%%%%%%%%%%%%%%%%%%%%%%%%%%%%%%%%%%%%%%%%%%%%%%%%%%%%%%%%%%% 
%% lstlinebgrd.sty
%% see: http://tex.stackexchange.com/questions/18969/creating-a-zebra-effect-using-listings/18989#18989
%% 
%% This small package is not yet published/not commonly available.
%% 
%%%%%%%%%%%%%%%%%%%%%%%%%%%%%%%%%%%%%%%%%%%%%%%%%%%%%%%%%%%%%%%%%%%%%%%%%%%%%% 

% Patch line number key to call line background macro
\lst@Key{numbers}{none}{%
  \def\lst@PlaceNumber{\lst@linebgrd}%
  \lstKV@SwitchCases{#1}%
  {none&\\%
    left&\def\lst@PlaceNumber{\llap{\normalfont
        \lst@numberstyle{\thelstnumber}\kern\lst@numbersep}\lst@linebgrd}\\%
    right&\def\lst@PlaceNumber{\rlap{\normalfont
        \kern\linewidth \kern\lst@numbersep
        \lst@numberstyle{\thelstnumber}}\lst@linebgrd}%
  }{\PackageError{Listings}{Numbers #1 unknown}\@ehc}}

% New keys
\lst@Key{linebackgroundcolor}{}{%
  \def\lst@linebgrdcolor{#1}%
}
\lst@Key{linebackgroundsep}{0pt}{%
  \def\lst@linebgrdsep{#1}%
}
\lst@Key{linebackgroundwidth}{\linewidth}{%
  \def\lst@linebgrdwidth{#1}%
}
\lst@Key{linebackgroundheight}{\ht\strutbox}{%
  \def\lst@linebgrdheight{#1}%
}
\lst@Key{linebackgrounddepth}{\dp\strutbox}{%
  \def\lst@linebgrddepth{#1}%
}
\lst@Key{linebackgroundcmd}{\color@block}{%
  \def\lst@linebgrdcmd{#1}%
}

% Line Background macro
\newcommand{\lst@linebgrd}{%
  \ifx\lst@linebgrdcolor\empty\else
  \rlap{%
    \lst@basicstyle
    \color{-.}% By default use the opposite (`-`) of the current color (`.`) as background
    \lst@linebgrdcolor{%
      \kern-\dimexpr\lst@linebgrdsep\relax%
      \lst@linebgrdcmd{\lst@linebgrdwidth}{\lst@linebgrdheight}{\lst@linebgrddepth}%
    }%
  }%
  \fi
}

\makeatother
