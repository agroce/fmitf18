\section{Test Normalization}

Normalization is the operation of rewriting a test (as always, subject
to some causal constraint) into a restricted form.  The constraints on
normalization are twofold:  a useful normalization must converge on a
``simpler'' test in some sense, and the steps of normalization must
have a low probability of drastically changing test behavior (inducing
slippage).  The potential benefits of normalization are substantial.  First, it
often rewrites many distinct failing tests for one fault into a single
failing test, making it easy to find the set of distinct faults in a
set of failing tests, a major problem in automated testing
\cite{PLDI13}.   Second, normalized tests are often both shorter and
include fewer features of the system under test, making them much
easier to use for debugging.

In preliminary experiments, we have produced a set of
rewrite rules for normalization in TSTL.  For a large set of real test
cases for real faults in the widely used SymPy Python symbolic math
library, normalization was able to reduce the mean number of failing
tests per fault from 12.5 tests to 3.15 tests.  The mean length and number of
distinct API calls were reduced, compared to delta-debugging by over
45\% and 35\%, respectively.  All results were statistically
significant with $p < 0.005$.  Figure \ref{lengthandactions} shows the
improvement in length and actions for the 39 SymPy faults discovered
using TSTL.  The {\bf reduced} plots show length and actions
statistics for tests for each fault, using only delta debugging.  The
{\bf normalized} plots show the same values, when test normalization
is also applied.  In addition to gains in test size and complexity
that ease debugging, the figure also partly shows the gain for fault
identification:  the ``boxes'' in this boxplot are typically single
lines for normalization because normalization tends to result in a set
of different tests exposing each fault being converted into a set of
identical tests, hence tests having the same length and action count.

\begin{figure}[t]
\includegraphics[width=\columnwidth]{sympyd}
\caption{Effects of normalization on SymPy tests.}
\label{lengthandactions}
\end{figure}

Normalization is combined with an inverse-operation, generalization,
that summarizes the set of similar tests that also satisfy some causal
contraint.  This lets a user have the advantage of examining a
simplified, terse test using only small input values, without any
disadvantage due to misleading information about the criticalit of
each aspect of the test.

At this stage, however, normalization is only defined for TSTL tests,
and has significant performance problems (it costs an order of
magnitude more time than delta-debugging, in many cases).  As part of
our effort to unify data and code-based tests, we propose to further
generalize the notion of normalization and generalization, provide
normalization facilities for key tests types such as C code (via
C-Reduce), and improve the effectiveness and efficiency of normalization.
