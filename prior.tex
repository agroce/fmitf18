%%%%%%%%%%%%%%%%%%%%%%%%%%%%%%%%%%%%%%%%%%%%%%%%%%%%%%%%%%%%%
\section{Results From Prior NSF Support}

The most relevant prior NSF support for PIs Groce and Regehr is
CCF-1217824, ``Diversity and Feedback in Random Testing for Systems
Software,'' with a total budget of \$491,280 from 9/2012 until 9/2015,
a collaborative proposal between the PIs.

\paragraph{Intellectual Merit}

The results of CCF-1217824 include a preliminary exploration of how to
``tame'' fuzzer output, a problem also considered in this proposal
\cite{PLDI13}.  In previous work, the goal was to find an algorithm
for using hand-chosen distance metrics to identify bugs in tests; in
this proposal, other methods for taming fuzzers are addressed.  A key related result from CCF-1217824
is the development of a strategy for creating very ``quick
tests'' from time-consuming randomized test suite by minimizing each
test with respect to its code coverage \cite{icst2014}, which won the
Best Paper award at the 2014 International Conference on Software
Testing. 
This work showed that tests reduced with respect to
code coverage can serve as effective regression tests or seeds for
symbolic execution \cite{stvrcausereduce,issta14}.  Moreover, we
showed that the benefits of such reduction do not depend on 100\%
preservation of a property, when that property is quantitative
(e.g. coverage) rather than qualitative \cite{ASEAdeq}.  Cause
reduction, with and without complete preservation of properties, is a
core component of our approach to test operations, and a primary
demonstration of our conceptual framework, where two tests that
satisfy the same key properties are assumed to be interchangeable.

CCF-1217824 also contributed to the design and development of the TSTL
tool \cite{NFM15,tstlsttt}, which supports fully automatic swarm
testing, based on the actions defined in a test harness, and which is
a core element of work for this proposal.

Other results from this project include an overview of the value of coverage in
testing experiments \cite{Onward14} and exploration of how individual
test features impact the coverage and fault detection statistics of
random tests \cite{helphelp}.  The basic swarm testing approach has
been extended to allow production of focused random tests targeting
particular code \cite{DirectedSwarm}.  At the broader level, CCF-1217824 has
produced a general set of results that focus on making automated random
testing usable by practitioners, and using symbolic execution on
larger, realistic software, in particular understanding how code
coverage and test content interact, either through minimization or
through statistical analysis.  Publications resulting from this grant
are numerous, including ones 
\cite{Onward14,PLDI13,issta14,icst2014,helphelp,DirectedSwarm,stvrcausereduce,tstlsttt,ISSTA15,ASEAdeq} cited above with relevant results.

\paragraph{Broader Impact}

The results of CCF-1217824 have been used in teaching software
engineering to undergraduates at Oregon State University. At
the University of Utah PI Regehr developed a new course ``Writing
Solid Code'' during this time period, that focused almost wholly
on software testing, based in part on research driven by this
grant.
%
Work done by both PIs has contributed to the discovery of previously
unknown faults in multiple open source and commercial software
systems.  The further development of the swarm testing techniques the
proposal centers on have in particular furthered the effort to improve
the quality of compilers, including LLVM and GCC, and to test language
tools in general\cite{ZhendongPLDI14,beginnerluck,dewey2015fuzzing,le2015randomized}.
Source code for tools resulting is available online at GitHub in the
{\tt swarmed\_tools} repository, and data (too large for hosting
services) is available upon request; TSTL is available on GitHub at
\url{https://github.com/agroce/tstl}, and key data from fuzzer taming
is available at \url{https://github.com/agroce/mutants16/tree/master/tests}.



%%%%%%%%%%%%%%%%%%%%%%%%%%%%%%%%%%%%%%%%%%%%%%%%%%%%%%%%%%%%%
