%%%%%%%%%%%%%%%%%%%%%%%%%%%%%%%%%%%%%%%%%%%%%%%%%%%%%%%%%%%%%
\begin{center}
{\Large\sf\textbf{Collaboration Plan --- 
CSR: Large: Collaborative Research:\\ \proptitle{}}}
\end{center}



%
Each year, most of the PIs see each other at PLDI (The ACM Conference
on Programming Language Design and Implementation) and attached
conferences/workshops.
%
In practice, it is common for us to see each other more frequently
than this, for example at program committee meetings, NSF panels,
and at other conferences.
%
We will exploit these opportunities to meet whenever possible, for
example by adding an extra day before or after an existing trip.
%
This is a highly cost-efficient way to talk face-to-face.
%
Additionally, the PIs at Utah all have offices on the same hallway and
see each other almost daily.


Our graduate students and our postdoc can be expected to travel less
frequently than the PIs, and so special effort will be required to
ensure that their efforts remain synchronized across institutions.
%
We propose the following:
\begin{compactenum}
\item
A monthly teleconference/Skype meeting attended by all
students and PIs.  Discussion will include status updates from each
institution, integration status and plans, and 
project dependences that need to be ironed out.
\item
Ad-hoc teleconference/Skype meetings as needed among subsets
of the PIs and students, especially during time periods leading up to
paper deadlines and code releases.
\item
An annual two- or three-day \name{} workshop, probably during
the summer, to help ensure that the postdoc and graduate students
remain aware of each others' progress and needs.
%
Most likely, we will alternate the workshop location to make it easier
for undergraduates and peripherally involved graduate students to
attend.
\end{compactenum}
%
Allocated carefully, we believe that our travel budgets will be
sufficient to enable these activities.


\paragraph{Management Across Investigators and Institutions}
%
Management of the proposed effort will be simplified by the fact that
three out of five PIs are at the University of Utah.
%
Thus, there is an obvious focal point for coordination efforts.
%
Our main priority, then, will be ensuring that work done at UVA and
Pitt remains synchronized with work that is happening at Utah.


\emph{We will exploit Lathe's modular structure to reduce dependences
  and make it easier to manage research across investigators and
  institutions.}
%
Figure~\ref{fig:newway} shows, and Section~\ref{sec:work} describes, a
collection of parts that, together, will enable applications for
ultra-low-power processors to be shaped to meet platform-specific
constraints and application-specific requirements.
%
We believe that most of the elements of the proposed infrastructure
can be developed and tested separately, using---when
necessary---simulated data and other mockups of missing pieces.
%
For example:
\begin{compactitem}
\item
Assume that Dr.~Hall and her student want to test a multi-objective
search based on the function inlining and outlining optimizations.
%
However, a predictive model of the effects of these optimizations,
that should be used to guide the search, does not yet exist.
%
Instead of the model, we can use real data from function inlining and
outlining trials, suitably degraded to match the expected quality of
the model.
\item
Assume that Dr.~Childers' group is working on modeling the effects
of scratchpad allocation, but the ARMv7 backend for LLVM is not robust
enough for serious use.
%
In this case, LLVM's very solid x86 backend could be used to compile
embedded codes and Cachegrind, a highly configurable cache profiler
and simulator built on Valgrind~\cite{Nethercote07}, used to provide
feedback to build the model and perform initial evaluation.
\end{compactitem}
%
By exploiting our closely aligned interests, our frequent in-person
contact, and the modularity of the proposed software system, we
believe the proposed efforts can be sufficiently synchronized.


\paragraph{Specific roles of the project participants in all organizations
involved}

We have requested funding for one student per year for each PI and
co-PI, in addition to one postdoctoral researcher who will reside at Utah.
%
In the table below, we describe six principals: the five PIs and
the postdoc.
%
We have not broken out separate tasks for graduate students funded
under the proposed grant, but rather we assume that the students and
their faculty advisors will be closely collaborating on the research
problems.
%
When multiple PIs are involved in a single effort, we indicate the
leader in a bold font.

\vskip 0.2in
\noindent
\begin{center}
\begin{tabular}{l|l|l}
\hline
\textbf{Year} & \textbf{Subtask Description} & \textbf{Participants}\\ \hline
\hline
\multicolumn{3}{c}{\textsc{Application and Platform Specification}}\\ \hline
Y1--Y2 & Metric Specification Language & \textbf{Regehr}, Might \\
Y1 & Specifying sensor network platforms & \textbf{Regehr}, Postdoc\\ 
Y2 & Metrics for sensor network applications & \textbf{Regehr}, Postdoc\\
\hline
\multicolumn{3}{c}{\textsc{Search}}\\ \hline
Y1 & Investigate search algorithms & \textbf{Hall}\\
Y1--Y5 & Feedback from external tools & \textbf{Childers}, Hall, Regehr, Postdoc\\ 
Y2--Y3 & Designing effective cost functions & \textbf{Regehr}, Hall\\
Y3--Y4 & Efficient global search & \textbf{Hall}\\
Y4 & Parallel search & \textbf{Hall}\\
\hline
\multicolumn{3}{c}{\textsc{Architecture Specification}}\\ \hline
Y1 & Specification language & \textbf{Davidson}, Might\\
Y2 & MSP430 & \textbf{Davidson}\\
Y3 & ARMv7 & \textbf{Davidson}\\ 
\hline
\multicolumn{3}{c}{\textsc{New Optimizations}}\\ \hline
Y1-Y2 & Scratchpad allocation & \textbf{Hall}, Childers, Regehr\\
Y2 & Outliner for LLVM & \textbf{Davidson}, Regehr\\
Y3--Y5 & Superoptimizer & \textbf{Davidson}, Regehr, Might\\
\hline
\multicolumn{3}{c}{\textsc{Specifying and Modeling Optimizations}}\\ \hline
Y1--Y3 & Labeling optimization opportunities & \textbf{Hall}, Might\\
Y2--Y3 & Inlining and Outlining & \textbf{Childers}, Regehr\\
Y4 & Superoptimization & \textbf{Childers}, Davidson\\
\hline
\multicolumn{3}{c}{\textsc{Assertion Resolution}}\\ \hline
Y1--Y2 & High-precision pointer analysis & \textbf{Might}\\
Y3--Y4 & Decision procedure integration & \textbf{Might}\\
\hline
\multicolumn{3}{c}{\textsc{LLVM Backend Development}}\\ \hline
Y1--Y2 & MSP430 testing and enhancement & \textbf{Postdoc}\\
Y2--Y3 & ARMv7 testing and enhancement & \textbf{Postdoc}\\
\hline
\multicolumn{3}{c}{\textsc{Lathe Deployment}}\\ \hline
Y3 & Initial deployment & \textbf{Regehr}, Postdoc\\
Y3--Y5 & Deployment evaluation and support & \textbf{Postdoc}, Regehr, Might\\
Y5 & Second deployment & \textbf{Postdoc}, Regehr\\
\hline
\end{tabular}
\end{center}


%%%%%%%%%%%%%%%%%%%%%%%%%%%%%%%%%%%%%%%%%%%%%%%%%%%%%%%%%%%%%
