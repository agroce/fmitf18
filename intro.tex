%%%%%%%%%%%%%%%%%%%%%%%%%%%%%%%%%%%%%%%%%%%%%%%%%%%%%%%%%%%%%

\section{Introduction}

Developing reliable networked systems whose final endpoints are
power-constrained, limited-bandwidth, and costly for humans to
repair or even inspect is a key challenge over the next decade; while the
widely discussed ``Internet of Things'' may be the first thing that
comes to mind given this description, the problem is also central to long-term
scientific research efforts, ranging from the climate change and
plant migration research platform described in this proposal to
efforts to understand the surfaces of other planets.

Formal methods have been used to verify aspects of such systems,
including file systems, control elements, network protocols,
\Fix{Let's find some cites and add things here}; however, the proofs
of components are seldom, if ever assembled into a coherent,
quantified and qualified understanding of the extent to which the
whole system has been verified.  Because different components of many
such systems are developed by disparate groups, sometimes
collaborating internationally, even when formal methods are uniformly
used, the emphasis is on the plural:  formal methods, not a single
formal method.  The assumptions of one formal method/component are
seldom formally yoked to the outputs of another formal method, with
gaps in verification made clear.   Finally, some components are only
verified using high-confidence, automated, but non-proof-based
approaches, including model checking with unsound abstractions, or
(model-driven) automated test generation.  When these kinds of
``semi-formal'' methods (relying on a formal model of inputs and
system behavior, but not providing correctness proofs) are integrated
into a design, there is almost never a characterization of the
confidence thus provided.  The outcome is either not used at all in
other methods, or is assumed to be as solid as a proof.

In this proposal, we aim to address this weakness by designing
``glue'' formal methods for assembling the results of a set of
hetergenous formal approaches into a single (if complex) result, able
to identify both weaknesses and strengths in a system verification.
E.g., if the proof of correctness of one component relies strongly on
an assumption backed only by limited, manually constructed tests, this
can be distinguished from the case where the assumption is backed by a
proof of correctness in a different formalism, which can be
distinguished from the case where it is backed by
both model-based and  symbolic-execution driven test generation, and
a probabilistic or coverage-based estimate of completeness is
available.  Moreover, the proposed approach will be able to distinguish the case
where an entire verification result is tainted by a non-proven
assumption from the (common) case where different properties or
behavioral paths have different degrees of reliabilty, ranging from
complete independence from an uncertain assumption to confidence
completely proportional to the strength of the assumption.
%%%%%%%%%%%%%%%%%%%%%%%%%%%%%%%%%%%%%%%%%%%%%%%%%%%%%%%%%%%%%


\subsection{Problem Statement and PI Qualifications}

\subsection{Expected Outcomes}

\section{Contributions to Formal Methods and the Field}