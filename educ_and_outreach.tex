%!TEX root = NSFMedium_TrustML.tex
%% \subsection{Education and Outreach}

The proposed research yields several opportunities for enhancing CS
education, recruiting new CS majors, and retaining CS students,
particularly members of underrepresented groups. Our education
activities directly relate to the following broader impact goals: (1)
Development of a globally competitive STEM workforce; (2)
Increased participation of women, persons with disabilities, and underrepresented minorities in STEM;
(3) Improved pre-K-12 STEM education; and (4) Improved undergraduate STEM education.


\subsection{Teaching Testing to Non-Computer-Scientists}

PI Regehr views testing as very under-emphasized in the typical
computer science curriculum and he makes testing an explicit area of
focus in every course that he teaches, whether it is compilers,
operating systems, or embedded software.
%
In the winter and spring of 2017 he organized an after-school ``coding
club'' for the 5th and 6th grade classes at Whittier Elementary, the
public school in Salt Lake City that his son attended, which is
located in a somewhat disadvantaged neighborhood (median annual
household income $\$35,446\pm\$11,944$, according to 2014 census
data).


He taught the students Python, starting with turtle graphics and using
that as the basis for teaching fundamentals like variables,
loops, functional abstraction, and recursion.
%
As the students grappled with these concepts, he also emphasized
methods for creating ``glitch free'' code (kids this age seem to have
an intuitive grasp of software bugs, especially in games, and call
them ``glitches''), such as trying to exercise corner cases in the
code and trying random inputs.


In spring 2018 PI Regehr will be teaching a software engineering
course, focused mainly on testing, as part of a new professional MS
program at the University of Utah.
%
The students will not have CS degrees, but rather come from diverse
backgrounds.
%
His course will emphasize the fundamental issues, that it hardly makes
sense to create software unless we have a precise understanding of
what it is supposed to do, and that once we have that understanding,
it can be operationalized using tests.



\subsection{Excursions in Testing}

PI Groce will be working with the NAU Student ACM Chapter to present a
series of ``excursions in testing'' that use automated testing to
introduce popular or exciting Python libraries to undergraduates.  In
particular, some of these excursions will focus on media-related
libraries or bioinformatics libraries.  The work of Guzdial
\cite{Guzdial} has shown that media computation is a
potentially effective way to both recruit and retain female and
under-represented minority students in computer science.
More than 55\% of biology majors are female; bioinformatics is a
bridge between STEM majors that are not lacking in female students and
those majors (such as CS) that continue to lag in recruitment and
retention of women.  Testing excursions introduce a Python topic and
allow students to participate in possibly finding and reporting bugs
in real software, which forms a strong connection to real, ongoing
open-source development efforts, in same cases even before students
have strong programming mastery.

\mycomment{

\subsection{Enhancing CS Undergraduate Education}

\mycomment{The OSU CS department already offers a required course on testing,
verification, and debugging, taught by PI Groce.  A major focus of
this class is on test automation and understanding the ``signatures''
of different bugs.  The ideas from this proposal will be integrated
into this class, where distance-based approaches to fault localization
are already discussed. As a motivating example, students will be taken
through a simplified version of the task of measuring distance between
two hypothetical telemetry streams from Mars Rover days, to understand
how distance between outputs can indicate problems with software.}

\subsection{Recruiting to the CS Major, Particularly Members of
  Underrepresented Groups}
}

\mycomment{
As part of this proposal, Dr. Groce at Oregon State University will
also partner with the Computer Science department's long-time
collaborator, Saturday Academy, to conduct an activity that teaches
the basic idea of comparing program executions, one faulty and one
successful, to high school students in computer science classes.  The
activity will be organized and publicized by Saturday Academy.

Saturday Academy is an award-winning cooperative, involving Oregon businesses,
professionals, and educators. It provides hands-on, after-school and
summer learning opportunities in science, technology, engineering, and
math for K-12 youth. Saturday Academy at OSU currently teaches 800
Oregon students annually, offering more depth than traditional schools
allow and helping students explore career opportunities. It serves
many communities in the Willamette Valley and beyond, including a
large Hispanic population, also reaching rural youth in small, remote
communities in Oregon, where few educational opportunities beyond
school are available. 

Saturday Academy target demographics include underrepresented ethnic
minorities, rural, low income, and/or female students of middle school
age. Saturday Academy has considerable experience with and access to
each of these groups. A significant proportion of participants in 2011's
camp were female, about 21\% were eligible for free and
reduced lunch, about 11\% came from underrepresented ethnic minority
groups, and about 57\% were female. To recruit students, Saturday
Academy will use their previously successful methods, including
advertising at the schools, website advertisements, e-mail
announcements, and posted advertising in the local areas.


There are several benefits of integrating our work with this
activity. First, it will provide a hands-on tool by which high school
students can begin to understand the idea of thinking about how two
runs of a computer program differ, using an example that simplifies a
scenario on board an actual Mars Rover, drawn from PI Groce's NASA
experience.  Second, by gaining this experience, we hope to recruit
some of these high school students into computer science. Third, even
for those students who are not attracted to the major, it will add to
the students' understanding of the fallibility of technology and the
role of human effort in making everyday tools trustworthy.
}

%% The following come from: http://nsf.gov/nsb/publications/2011/06_mrtf.jsp
%
% National Goals:
%    * Increased economic competitiveness of the United States.
%    * Development of a globally competitive STEM workforce.
%    * Increased participation of women, persons with disabilities, and underrepresented minorities in STEM.
%    * Increased partnerships between academia and industry.
%    * Improved pre-KÐ12 STEM education and teacher development.
%    * Improved undergraduate STEM education.
%    * Increased public scientific literacy and public engagement with science and technology.
%    * Increased national security.
%    * Enhanced infrastructure for research and education, including facilities, instrumentation, networks and partnerships.
%
% Review criteria:
% 1. Which national goal (or goals) is (or are) addressed in this
% proposal? Has the PI presented a compelling description of how the
% project or the PI will advance that goal(s)?
% 2. Is there a well-reasoned plan for the proposed activities,
% including, if appropriate, department-level or institutional engagement?
% 3. Is the rationale for choosing the approach well-justified? Have
% any innovations been incorporated?
% 4. How well qualified is the individual, team, or institution to
% carry out the proposed broader impacts activities?
% 5. Are there adequate resources available to the PI or institution to
% carry out the proposed activities?

