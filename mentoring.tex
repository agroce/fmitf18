\begin{center}
{\Large\sf\textbf{Postdoctoral Researcher Mentoring Plan ---\\
CSR: Large: Collaborative Research: \proptitle{}}}
\end{center}

% The GPG says:
%
% \begin{quote}
%
% Postdoctoral Researcher Mentoring Plan. Each proposal that
% requests funding to support postdoctoral researchers must include, as
% a supplementary document, a description of the mentoring activities
% that will be provided for such individuals. In no more than one page,
% the mentoring plan must describe the mentoring that will be provided
% to all postdoctoral researchers supported by the project, irrespective
% of whether they reside at the submitting organization, any subawardee
% organization, or at any organization participating in a simultaneously
% submitted collaborative project. Proposers are advised that the
% mentoring plan may not be used to circumvent the 15-page project
% description limitation. See GPG Chapter II.D.4 for additional
% information on collaborative proposals
%
% Examples of mentoring activities include, but are not limited to:
% career counseling; training in preparation of grant proposals,
% publications and presentations; guidance on ways to improve teaching
% and mentoring skills; guidance on how to effectively collaborate with
% researchers from diverse backgrounds and disciplinary areas; and
% training in responsible professional practices. The proposed mentoring
% activities will be evaluated as part of the merit review process under
% the Foundation's broader impacts merit review criterion. Proposals
% that include funding to support postdoctoral researchers, and, do not
% include the requisite mentoring plan will be returned without review
% (see GPG Chapter IV.B.)
%
% \end{quote}


We are requesting funds to support one postdoctoral researcher during
each year of the proposed effort.
%
Since the ideal length of a postdoctoral post is probably less than
five years, we will most likely hire two different people: one
staying with us for two years, the other for three years.
%
The postdoc will be based at the University of Utah, where she will
have the opportunity to interact with the three Utah PIs as well as
several other faculty whose research intersects with the proposed
work: Dr.~Matthew Flatt (programming languages), Dr.~Ganesh
Gopalakrishnan (formal methods), and Dr.~Alan Davis (embedded
systems).


Dr.~Regehr will take primary responsibility for mentoring the
postdoc, with support from Dr.~Hall, who has previously supported and
mentored two postdocs.
%
Our mentoring efforts will include the following specific efforts:
%
\begin{compactitem}
%
\item
%
Although most courses in the School of Computing are taught by regular
faculty, there are opportunities for visiting and junior researchers
to teach an occasional course (for example, Dr.~Regehr taught the
undergraduate Operating Systems course as a postdoc in 2002).
%
\item
%
The School of Computing at Utah tends to support the
development of grant-writing among junior researchers. 
%
For example, Dr.~Regehr was a co-PI on an NSF grant that was authored
while he was a postdoc under Jay Lepreau's supervision.
%
Also, Dr.~Regehr is currently the PI on several grants that have
non-faculty co-PIs.
%
It is common for senior faculty members in the School of Computing to
read and critique drafts of grant proposals written by junior
researchers, and our front office and OSP provide solid support on the
bureaucratic aspects of grant-writing.
%
If circumstances warrant it, we will seek out a research faculty
appointment for the postdoc; this confers the ability to be a PI on
grants (postdocs can be co-PIs only when special permission is
obtained).
%
\item
%
PIs Hall and Regehr serve on many program committees and in fact both
have recently served as Program Committee Chairs for conferences in
their areas.
%
We will exploit our connections inside the compiler and embedded
communities to secure invitations to program committees for our
postdoc---it is often the case that postdoc-level PC members are in
high demand because they are in a sweet spot where they are
experienced enough to provide valuable evaluations, but they are not
yet overloaded with service work.
%
These memberships facilitate tight professional interaction with
leaders in the community, which are important for junior researchers.
%
\item
%
Dr.~Regehr will meet weekly or bi-weekly, as appropriate, for one hour
with the postdoc to discuss progress and future plans in
implementation, paper writing, and career development.
%
\item
%
At any given time there are around three reading groups in Utah's
School of Computing in areas such as compilers, systems, and program
analysis that are relevant to our postdoc's work.
%
We will encourage the postdoc to attend some of these and to give
presentations.
%
\item
%
The postdoc will be encouraged to attend relevant conferences and to
present papers in which she performed a leading role.
%
\item
%
Annually, the five PIs will discuss the postdoc's progress and
potential future directions, and provide the postdoc with a short
written report describing the meeting's outcomes.
%
\end{compactitem}

In summary, we believe that hands-on mentoring by Regehr and Hall,
augmented by a rich collection of ongoing research projects at Utah in
closely related areas, will provide a stable and nurturing environment
for our postdocs' career development.
