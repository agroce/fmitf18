%%%%%%%%%%%%%%%%%%%%%%%%%%%%%%%%%%%%%%%%%%%%%%%%%%%%%%%%%%%%%
\begin{center}
{\Large\sf\textbf{Data Management Plan\\
SHF:Small:Collaborative Research:\proptitle{}}}
\end{center}

The data in the proposed project is primarily of two kinds:
\begin{itemize}
\item
source code for testing tools
\item empirical data about testing experiments.
\end{itemize}

In the first case, TSTL and C-Reduce are already open source systems
hosted on GitHub.  

Curricular
materials associated with the testing tools will also be stored in
an open source repository, since in this project the primary
educational benefits are linked to the use of testing tools.  Using GitHub automatically provides us with excellent backup
and archiving for the code and curricular material products of the
project.

Empirical testing data will often be ephemeral and reproducible by
re-running experiments with known seeds.  If the computational burden
to recreate data is excessively large we will store data in a
compressed format on the open-source repository.
%
Given the ever-increasing power of computers, we do not expect
this to be the case for most data sets.


We have no unusual format or metadata requirements; test cases
themselves are (in our setting) ususally source code files or text
files readable by the tools, and testing results are standard
formats produced by instrumentation tools or stored in a format such
as CSV or XML files.

We do not anticipate the need to work with sensitive or confidential
information; no extraordinary practices are required in order to
conduct this research.
