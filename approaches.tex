%%%%%%%%%%%%%%%%%%%%%%%%%%%%%%%%%%%%%%%%%%%%%%%%%%%%%%%%%%%%%
\section{Using Modern Machine Learning for Effective Metrics}
Metric learning is one of the most fundamental problems in machine
learning. The goal of metric learning is to learn a distance function
that is tuned according the specifics of a given task
\cite{kulis2012metric}. While there exist some unsupervised
approaches (e.g., principal components analysis and its probabilistic
extensions \cite{scholkopf1998nonlinear,tipping1999probabilistic}),
most metric learning work focuses on learning from supervised examples
so that examples of the same class will be close to each other,
whereas examples of different classes are far apart in the learned
metric space.

For the sake of clarity and concreteness, we will initially focus on
linear models for metric learning in our discussion. Specifically,
consider the following standard setup for metric learning. We are
given a set of data points $\{x_1, x_2, ...,x_n\}$, where $x_i \in
R^d$ and $d$ is the original dimension of the data, which can be very
large. Our goal is to learn a Mahalanobis distance in the following
form: \[d_A(x_i, x_j) = (x_i-x_j)^TA(x_i-x_j),\] where $A$ is a
$d\times d$ positive semi-definite matrix (that is its eigen-values
are non-negative), such that $d_A(x_i, x_j)$ satisfy some
constraints. The constraints can be pairwise (specifying a pair to be
similar or dissimilar)
\cite{xing2002distance,bilenko2004integrating,davis2007information},
or relative (specifying a pair to be more similar than
another)\cite{schultz2004learning,rosales2006learning}.  A basic
principle used by many metric learning methods is to combine the
constraints on the distances with some form of regularization on
$A$. A variety of regularization terms have been considered in this
context to avoid overfitting, including the well known Squared
Frobenius norm (a matrix version of the popular $L_2$ regularizer)
\cite{schultz2004learning,kwok2003learning}, the trace norm
regularizer (analogous to the $L_1$ regularizer to encourage sparsity)
\cite{jain2010inductive}, and information-theoretic regularizers
\cite{davis2007information}. These metric learning methods have been
highly successful in many applications, including vision
\cite{hoi2006learning,frome2007learning,guillaumin2009you}, text
\cite{davis2008structured}, bioinformatics\cite{xiong2006kernel}, and
most relevantly, automatic program error reporting, where a metric is
learned to measure distance between program executions
\cite{davis2007information,HaClarify}.

Despite the availability of many successful metric learning
algorithms, the problems that we aim to address in this proposal
cannot be solved by simply applying these metric learning
techniques. This is mainly due to the fact that the standard
supervised information that we can acquire for many of our problems
are not sufficient for effective metric learning, or our desire to
learn (mixes of) Levenshtein metrics \cite{lev}.  Consider fuzzer
taming as an example: each $x$ represents a bug-inducing random test
case. We are given a large collection of bug-inducing random test
cases, and our goal is to learn a metric such that the test cases
inducing the same bug are deemed similar to one another, and test
cases inducing different bugs are dissimilar. Although this appears to
be a simple and rather direct application of existing metric learning
techniques, generally we do not have any prior knowledge about what
bugs these test cases induce and which ones are the
same/different. Consequently we are faced with the challenge that that
we would like to assist users even when we do not have sufficient
labeled data for traditional metric learning.

To address this challenge, our proposed approaches will consider two
types of supervision: the traditional example-level supervision that
captures local information between examples, and a novel form of
global task-level supervision that is informative of the success level
of a metric in performing the specific task at hand.  For local
example-level supervision, due to its limited availability, our
proposed approach will leverage the complex structure of our
examples, and draw on a separate body of literature on learning with
scarce data to enhance the effectiveness of metric learning. For
global task-level supervision, we propose to derive global
task-oriented effectiveness measures for learning metrics using
benchmarked tasks. Casting the metric learning problem as a black-box
function optimization problem, we will develop our solution by
exploiting some recent developments in Bayesian Optimization.

\subsection{Learning from Local Instance-level Supervision}
In this section, we will first explain how the complex structure of
the examples in our application can be leveraged to automatically
produce supervised information for learning metrics without involving
human effort.  The availability of such supervision can be highly
beneficial to learning but is still limited and sometimes can be of
only one particular type. We will then outline our research plan to explore
semi-supervised methods to learn with positive and unlabeled data.

\paragraph{Automatically generating example-level supervision.}
\label{sec:autosuper}

\begin{figure*}[t]
\centering
     \subfigure[Test case features, first 150 tests]{ \includegraphics[width=0.49\textwidth]{figures/fmin_discovery_detailed} }
     \hspace{-.6cm}
     \subfigure[Test case features, all tests]{ \includegraphics[width=0.49\textwidth]{figures/fmin_discovery} }
     \subfigure[Function coverage features, first 150 tests]{ \includegraphics[width=0.49\textwidth]{figures/fcov_discovery_detailed} }
     \hspace{-.6cm}
     \subfigure[Function coverage features, all tests]{ \includegraphics[width=0.49\textwidth]{figures/fcov_discovery} }
%     \subfigure[Test case features Subsample, first 50 tests]{ \includegraphics[width=0.45\textwidth]{figures/fmin_subsamp_discovery_detailed} }
%     \subfigure[Test case features Subsample, all tests]{ \includegraphics[width=0.45\textwidth]{figures/fmin_subsamp_discovery} }
\caption{\label{fig:disc} Bug discovery curve as a function of examined test cases using different features. {\it Single} and {\it Average} use
sets of reduced test cases to compute distances, while {\it reduced} only use the reduced minimal test case. For {\it Single}, {\it Average}, and {\it reduced} furthest point first is applied to find the ranking. }
\end{figure*}

For some of our tasks, it is possible to leverage the complex
structure of the examples to produce useful similarity
supervisions. Consider the task of fuzzer taming, where each example
is a bug-inducing random test case, which has been reduced using
delta-debugging \cite{DD} to remove noise.  Interestingly, the process
of delta-debugging actually produces a trail of test cases that are
known (or at least extremely likely) to trigger the same bug
\cite{PLDI13}. One basic hypothesis is that by considering the full or
partial trail of test cases generated by delta-debugging, we may be
able to define or learn better distance metrics. This hypothesis has
been partially verified by our preliminary investigation
\cite{DDTrail}, in which we treat a partial trail of test cases
generated by delta-debugging collectively as a single example and use
set-based distance metrics. The results show that we were able to
significantly improve the bug identification efficiency especially in
the early stages. Figure \ref{fig:disc} shows discovery curves based
on 1) using \emph{random} ordering as a baseline, 2) using just the
smallest \emph{reduced} test case (ignoring the delta-debugging trail)
3) using \emph{average} pairwise distances between sets and 4) using
\emph{single} distances between sets based on nearest pairs.  The
number of discovered classes in the critical early part of the curve
improves fastest for both coverage and test-case text feature vectors
when using the information from delta-debugging, though for test case
features single-linkage is not as good as using a single test case.
In this project, we will push this idea further by producing must-link
constraints between test cases of the same delta-debugging trail (as
we are generally certain that they all trigger the same bug) to help
learn a better metric. We believe this, in combination with the
set-based distances explored in our preliminary work \cite{DDTrail},
can further improve the efficiency of bug discovery.  While at first
appearance, using delta-debugging trails seems to be limited in
application to fuzzer taming, we also hope to take advantage of our
recent generalization of delta-debugging to other criteria (e.g. code
coverage) \cite{icst2014} to apply similar methods to test-case
generation as well, as discussed in Section \ref{sec:gen}.

\paragraph{Metric learning with positive and unlabeled data}
One issue with automatically generated similarity constraints is that
the amount of supervision is limited not only in terms of its
quantities, but also in terms of the type of information
provided. Consider such an automatically induced similarity
constraint, which indicates that a pair of instances should be deemed
similar to one another by the learned metric. This can be viewed as
labeling a pair of examples as similar, which we can interpret as a
\emph{positive} example. In fact all of the constraints that we
introduce in this fashion will be positive. In supervised learning,
one branch of research is to study how to most effectively learn from
Positive and Unlabeled data, this is referred to as PU learning
\cite{li2003learning, elkan2008learning}. PU learning has been
demonstrated to improve learning given limited supervision for text
classification\cite{li2003learning}, gene regulatory network learning
\cite{cerulo2010learning}, and remote sensing \cite{li2011positive},
but has not been applied in the context of metric learning. In this
research we will explore different PU learning approaches for metric
learning to further enhance the ability to learn appropriate metrics
from limited supervision.

\subsection{Learning from Global Task-level Supervision}
Consider the following form of supervision: we have a specific task
(e.g., fuzzer taming, where our goal is to identify as many bugs as
possible by examining only a small number of test cases). Given a
metric, we can apply this metric to a benchmarked task and measure its
performance. For fuzzer taming, this means to apply FPF with the
provided metric on a benchmarked dataset, and measure the area under the
bug discovery curve. The larger the area, the more effective the
metric is. Typically this type of measure is only used in the final
evaluation stage. We hypothesize that this performance measure can
also be effectively used as feedback and supervision to help improve
our metric.

Specifically, we propose to cast the metric learning problem into a
black-box function optimization problem \cite{jones2001taxonomy},
where the task-level performance measure is the black-box
function. Note that we can not directly optimize this function using
standard optimization techniques such as gradient-based methods,
because we do not know the form of this function. Instead, we can
empirically sample this function repeatedly and use the collected
information to search for the optimizing distance metric.

Bayesian Optimization (BO) is a successful Bayesian approach for
iteratively optimizing a black-box function $f$. In Bayesian
Optimization, we model the posterior distribution of the unknown
function $f$ (e.g., using Gaussian Processes) based on our prior
belief about $f$ as well as the set of currently observed samples. The
model is used to incrementally select new input configurations to
sample the function $f$, and the observed function values are then
used to update the model. BO has been successfully applied to a large
variety of applications including sensor
placement~\cite{krause2008near}, optimizing the anode design of
microbial fuel cells \cite{azimi2010myopic}, and tuning parameters of
large scale machine learning models to optimize the learning
performance \cite{snoek2012practical}. To the best of our knowledge,
BO has not been applied to optimize the performance of a distance
metric.  This is probably due to fact that distance metric learning
typically involves a large number of dimensions. For example, when we
represent test cases with function coverage, we can have thousands or
even tens of thousands of features. So the number of variables for the
metric learning problem is very large, exceeding the capability of
most BO algorithms, which can only handle variables at best. How can
we address this? Some recent development in BO points to some
potentially fruitful directions \cite{Wang:2013b}. Specifically,
random embedding has been demonstrated to be a promising mechanism for
reducing the dimensionality of the problem and allowing more
traditional BO methods to be effectively applied. We will investigate
the effectiveness of BO with random embedding for metric
learning. Unlike the example-level supervision, this framework is very
general and can be used with any task that uses a distance metric and
has a well defined performance measure and some benchmark datasets. We
believe that testing is a very promising domain for such an approach,
as there is often a set of, e.g., existing test cases that have been
evaluated for code coverage and fault detection, or known bugs to
train on, even though most of the testing work is in the future.
Large software projects also seem to evolve in a continuous enough
fashion that metrics may remain useful over a period of months or even
years; certainly individual test cases have a long
lifetime and persistent identity, including approximate code coverage \cite{icst2014}.

Finally, BO provides a promising approach for tuning metrics that are
not based on some function of a feature vector, but instead rely on,
for example, tuning costs of operations in a Levenshtein \cite{lev}
metric.  This is potentially critical, as some of the most successful
metrics in previous work have been Levenshtein-based rather than
feature-vector based \cite{ChakiLev,PLDI13}.  In addition to tuning
costs of individual operations or alignments, BO can also help best
combine multiple edit distances (e.g. over output and test text, or
execution sequences) into a single measure, or pick abstraction levels
to apply when actual edit distance over raw data is too expensive
(e.g. for call sequences).
