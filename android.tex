%%%%%%%%%%%%%%%%%%%%%%%%%%%%%%%%%%%%%%%%%%%%%%%%%%%%%%%%%%%%%
\section{Testing Android}

%%%%%%%%%%%%%%%%%%%%%%%%%%%%%%%%%%%%%%%%%%%%%%%%%%%%%%%%%%%%%

Sections~\ref{sec:advswarm} and~\ref{sec:symbolic} describe the PIs'
proposal to perform fundamental research in software testing.
%
This section describes how the testing research will be translated
into pragmatic, open-source tools that will help improve the quality
of some components of the Android operating system for mobile devices.
%
The research will proceed in three steps:
%
\begin{enumerate}
%
\item 
%
We will develop two new random testers for elements of the
  Android system.
%
\item
%
We will apply Swarm testing ideas to these testers.
%
\item 
%
We will use our new random testers to find Android bugs. This
  will permit us to concurrently evaluate our research ideas and to
  discover and report previously unknown Android defects.
%
\end{enumerate}


\subsection{Android}

Android is a Linux-based operating system for mobile devices.

\paragraph{Kernel Level}

It uses a modified kernel that includes:
%
\begin{itemize}
\item
many new device drivers specific to mobile devices
\item 
the \emph{binder}---a sophisticated and relatively complex
  subsystem implementing security, service lookup, and fast IPC
  (inter-process communication)
\item
a power management subsystem
\item
a more sophisticated ``low memory killer'' than the one included
with Linux
\end{itemize}



\paragraph{Middleware Level}

Much of Android's user-level code executes in the context of the
Dalvik Virtual Machine.
%
Dalvik uses a custom bytecode format that provides significantly
better code density than does Java's.
%
At boot time, Android creates a ``zygote'' VM that loads a set of
commonly-used libraries.
%
Whenever a VM is needed, it is forked from the zygote; Linux's
copy-on-write fork implementation in combination with clever data
layout tricks (particularly involving the garbage collector) minimize
the amount of data that will be written.
%
Thus, Dalvik sidesteps the slow startup times commonly
associated with JVMs\@.


Android runs a customized graphics stack, eschewing the X Windows
system commonly found on Linux machines.
%
High-performance graphics applications use OpenGL bindings for 
Java, C, or C++.


\paragraph{Application Level}

Android applications are written in Java, which is first compiled to
Java bytecodes and then into the Dalvik format.
%
Applications make use of a class hierarchy providing access to 
GUI features, sensors, network functionality, etc.
%
The basic unit of application programming is the \emph{activity},
which interacts with a user via buttons, dialogs, etc.
%
Activities have a well-defined lifecycle, and events are delivered for
example when the activity is created or destroyed, becomes visible or
hidden, becomes the foreground activity or not, etc.
%
The system maintains an \emph{activity stack} that users can navigate
through using dedicated buttons on the tablet or phone.
%
Android activities are complemented by \emph{services}, which
implement long-running computations and do not interact directly with
users.


Activities in Android can invoke other activities from the same
or from different applications using \emph{intents}.
%
Intents serve as a mechanism/policy boundary, as a mechanism for late
binding, and also as a boundary for security policy enforcement.
%
For example, an application would use an intent to share a
photograph on the Internet.
%
The Android system can choose an appropriate service (e.g., Facebook),
it can fail if no appropriate service is registered, or it can ask the
user which of multiple appropriate services to use.


Unlike typical UNIX or Windows applications, which execute under the
full set of privileges available to the user running the application,
Android applications are sandboxed.
%
First, every application is given its own UNIX userid, preventing
it from accessing files created by other applications.
%
Second, at install time, an application is granted a collection of
permissions such as ``access fine-grained location data,'' ``access
local storage,'' ``send SMS messages,'' and ``use Bluetooth.''
%
Android applications are sandboxed at the level of the Linux
process, not at the level of the Dalvik VM\@.
%
The distinction is important since Android applications are permitted
to run native code libraries that can subvert Java-level guarantees.


\subsection{Testing Android Applications}

We will develop a domain-specific random tester for Android
applications.
%
It will focus on three aspects of the Android API that were
described above: activities, intents, and services.
%
In all three cases, the Android code is event-driven; our random
tester will fire random events using a collection of state machines to
determine which events are allowable.


We expect to find bugs in activities because they must implement
a fairly complicated state machine.
%
In fact, we have seen cases where open-source Android applications
placed code in the wrong event handler; this code worked fine in the
expected case but can cause a resource leak or user-visible
malfunction when events arrive in an unexpected (but allowed) order.


Since intents are used as gateways between applications with different
permissions and levels of trust, bugs in code implementing intents can
lead to exploitable vulnerabilities.
%
Thus, random testing of these interfaces is important.


Finally, we will perform random testing of the service interface.
%
Code in services runs, by default, in the application thread.
%
However, in most cases services are designed to run in the background,
so they generally fork off one or more threads to perform activities
such as downloading or uploading a large file, playing music, etc.
%
Multithreaded code is often a fruitful place to look for bugs, and in
fact we have seen open-source Android applications that fork off more
threads than are strictly needed, and that have excessively
complicated communication protocols between threads.
%
These are smoking guns for serious concurrency errors.


\paragraph{Oracles}

Random testing only makes sense in the presence of one or more
appropriate oracles.
%
Since high-level oracles---executable specifications, or alternate
implementations---are not expected to be available, our testing
activity will focus on \emph{low-level oracles}.
%
First, our random testing infrastructure will assert a number of
interface invariants that we learn from the Android documentation.
%
In particular, adhering to the activity state
machine\footnote{\url{http://developer.android.com/reference/android/app/Activity.html}}
is difficult.
%
Second, any time an application exits with an uncaught exception, a bug
has likely been found.
%
Third, TaintDroid~\cite{Enck10} is a dynamic dataflow tracking system
that can be configured to issue a warning when private data (email,
contacts, location, etc.) is leaked out over a network interface.
%
Because the Android permission system is quite coarse-grained, many
applications have the capability to leak private information, but they
are not expected to actually do so.
%
A number of race condition detectors for multi-threaded Java applications
have been developed~\cite{Elmas07,Flanagan09}.
%
We will port a Java race detector to Android (if nobody else does
this first); data races are fairly reliable indications of logic
errors in application code.


\subsection{Testing Android Kernel Modules}

Linux kernel modules (LKMs) are binary objects that are dynamically
loaded into the kernel at runtime.
%
Each LKM requires that certain symbols are defined and optionally
defines more symbols that are exported for subsequent modules to load
against.
%
In this fashion, stacks of device drivers can be built.


Our belief is that LKMs are best tested in their binary form.
%
To do this, we will implement a user-mode loader for LKMs; 
this is just a re-implementation of the kernel's LKM loader (and in
fact it can probably borrow code from the kernel).
%
At this point, the random testing infrastructure will issue calls
to the driver, exercising its functionality.
%
The advantage of testing binary device drivers in user mode is that we
are testing the actual code that runs in deployed systems.
%
This is useful because sometimes, interactions with the compiler used
to build the kernel lead to exploitable
bugs\footnote{\url{http://www.theregister.co.uk/2009/07/17/linux_kernel_exploit/}}
these bugs could be masked if we used a different compiler, or
even different flags.


The disadvantage of our proposed approach---compared with, for
example, exercising drivers ``in place'' inside a running kernel---is
that it places a significant implementation burden on us, to implement
emulated versions of each kernel subsystem with which the driver
interacts.
%
We will cope with this problem by focusing our testing efforts
on a few specific families of device drivers.
%
Each member of a device driver family has similar interfaces but
different internal logic.
%
For example, there is a family of Ethernet drivers, sound card
drivers, filesystem drivers, etc.
%
The work required to perform aggressive random testing of a single
family of device drivers will be modest.
%
In fact, PI Groce has significant experience in performing random
testing of filesystems---an important family of device drivers.


The drivers that we focus on will be Android-specific, such as those
interfacing with the touch screen, GPS, and cellular modem.
%
Our belief is that these drivers are relatively poorly tested
compared to mainstream Linux device drivers.
%
We will also test wireless network drivers since bugs in these can
permit remote exploits.\footnote{An excellent list of known Linux
  vulnerabilities (including many that are remotely exploitable) can
  be found here:\\ \url{http://www.exploit-db.com/platform/?p=linux}}


\paragraph{Oracles}

First, domain-specific assertions will serve as test oracles.
%
For example, a driver that is in a ``stopped'' state should not issue
events to operating system.
%
Second, device drivers are written in C and oracles for C code abound.
%
Our own integer overflow checker\footnote{Open source software found
  here: \url{http://embed.cs.utah.edu/ioc/}} (IOC) has detected a
large number of potentially harmful integer overflow bugs in C
programs and we expect it to continue to be useful for testing kernel
code.
%
Valgrind, the Google address sanitizer, and many other tools
are available to detect memory safety errors in C code; we will
use them to find driver bugs.


\subsection{Swarm Engineering for Android Apps and LKMs}

Our early experience with swarm is that random testers already support
(i.e., for other reasons) the ability to turn off many kinds of
features, usually via the command line or a configuration file.
%
Some aspects of swarm testing can be completely factored out of the
random test case generator; for example, if a feedback mechanism wants
to use a Hamming-distance mechanism to explore configurations ``near''
ones that have already found a lot of bugs, this requires no special
support in the test case generator.
%
On the other hand, other aspects of swarm, such as parameterizing test
generators with probability distributions, requires them to be
modified.
%
As part of the proposed research we will attempt to determine how to
best engineer test case generators that can gracefully plug into a
swarm testing framework.



%%%%%%%%%%%%%%%%%%%%%%%%%%%%%%%%%%%%%%%%%%%%%%%%%%%%%%%%%%%%%


