\documentclass[11pt]{article}
% ----------------------------------------
\usepackage[numbers]{natbib}
\usepackage[in]{fullpage}
\usepackage{doi}
\usepackage{xspace}
\usepackage{hyperref}
% ----------------------------------------
\newcommand{\framac}{\textsc{Frama-C}\xspace}
\newcommand{\pathcrawler}{\textsc{Path\-Craw\-ler}\xspace}
\newcommand{\acsl}{\textsc{Acsl}\xspace}
\newcommand{\eacsl}{\textsc{E-Acsl}\xspace}
\newcommand{\eacsltoc}{\textsc{E-Acsl2c}\xspace}
\newcommand{\Wp}{\textsc{Wp}\xspace}
\newcommand{\Dafny}{\textsc{Dafny}\xspace}
\newcommand{\Coq}{\textsc{Coq}\xspace}
\newcommand{\Value}{\textsc{Va\-lue}\xspace}
\newcommand{\Eva}{\textsc{Eva}\xspace}
\newcommand{\eva}{\Eva}
% ----------------------------------------
\begin{document}

While the correctness of an implementation with respect to a formal
functional specification provides a very strong form of guarantee, it
can be very costly to achieve, and is currently mostly reserved to
domains where it is required by regulations or offers a competitive
advantage. In practice, it is very useful to rely on a combination of
formal methods to achieve an appropriate degree of guarantee:
automatic static analysis to ensure the absence of runtime errors,
deductive verification to prove functional correctness, and runtime
verification for parts of code that cannot be (or are not yet) proved
using deductive verification, or parts of code that contains warnings
detected by static analysis that need further analysis to determine
whether the warning is an error or not.

This project will use \framac{}\footnote{See
  https://frama-c.com}~\cite{KKP2015:FAC} is a source code analysis
platform that aims at conducting verification of industrial-size
programs written in ISO C99 source code. \framac{} fully supports
combinations of different approaches, by providing its users with a
collection of plugins for static and dynamic analyses of safety- and
security-critical software. Moreover, collaborative verification
across cooperating plugins is enabled by their integration on top of a
shared kernel, and their compliance to a common specification
language: \acsl~\cite{ACSL}.

\acsl, for ANSI/ISO C Specification Language, is based on the notion
of contract like in JML. It allows users to specify functional
properties of programs through pre/post-condition, and provides
different ways to define predicates and logic functions.  Some useful
built-in predicates and logic functions are provided, to handle for
example pointer validity or separation.

\emph{Value analysis} is a program analysis technique that computes a
set of possible values for every program variable at each program
point.  It is based on the \emph{abstract interpretation} technique
proposed by Cousot and Cousot in the 1970's~\cite{cousot77}.  Its main
idea is to compute an abstract view of values of variables in the form
of \emph{abstract domains}. For example, a usual abstract view for a
number value is an interval.

Value analysis can be very useful to detect potential runtime errors
or prove their absence.  Typical examples include invalid pointers,
invalid array indices, arithmetic overflows or division by zero.  It
can also help to prove other properties for which domain-based
reasoning can be efficient.

Since the \framac Aluminium release, \framac offers a new value
analysis plugin \Eva (Evolved Value Analysis)~\cite{eva-manual}.  It
implements value analysis as a generic extendable analysis
parameterized by cooperating abstract domains.  Different, highly
optimized domains are used to represent integers, floating-point
numbers and pointers.  \Eva is strongly integrated into the \framac
ecosystem and offers a basis for many other derived plugins that reuse
the results of \Eva (see~\cite{KKP2015:FAC}).

\Wp is a \emph{deductive verification} plugin provided with
\framac. It is based on weakest precondition calculus. Given a C
program annotated in \acsl, \Wp generates the corresponding proof
obligations that can be discharged by SMT solvers or with interactive
proof.  A combination of automatic and interactive proofs often offers
a good trade-off for a complete proof. Indeed, some properties can
only be defined recursively, and in this case, SMT solvers often
become inefficient, trying to unroll them. By using inductive or
axiomatically defined functions, we can prevent this behavior but
reasoning about them still requires induction, a task that SMT solvers
are not good at. Thus, the last step is generally to state lemmas that
can be directly instantiated by SMT solvers.

\framac was initially designed as a static analysis platform, but it
was later extended with plugins for dynamic analysis. One of these
plugins is \eacsl, a runtime verification tool.

\eacsl supports runtime assertion
checking~\cite{CR2006:SEN}. Assertions are very convenient for
detecting errors and providing information about their locations.  It
is the case even when such an error does not result in a failure
during execution. 

In \framac, \eacsl is both the name of the assertion language and the
name of a plugin that generates C code to check these assertions at
runtime. For the sake of clarity from now on we will use \eacsl only
for the language, and \eacsltoc for the plugin.

\eacsl is a subset of \acsl: the specifications written in this subset
can therefore be used both by \Wp and \eacsltoc. \Wp tries to prove
the correctness of these assertions {\em statically} using automated
provers, while \eacsltoc is used to translate these assertions into C
code that can then be executed. In this case the assertions are
checked {\em dynamically}.

\bibliographystyle{plainnat}
\bibliography{bibliography_fl}

\end{document}


